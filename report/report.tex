\documentclass{article}
\usepackage[final]{nips_2018}
\usepackage[utf8]{inputenc}
\usepackage[T1]{fontenc}
\usepackage{hyperref}
\usepackage{url}
\usepackage{booktabs}
\usepackage{amsfonts}
\usepackage{nicefrac}
\usepackage{microtype}

\title{Learning to Grasp: From the Cornell Dataset}
%\author{George Maratos (need to mention advisor)}

\begin{document}
\maketitle

\begin{abstract}
Fully autonomous grasping is a difficult problem. In this project, I explored
the cornell grasping dataset and tried to see if I could build a model that
learning how to model the information from this. The task to be learned was
difficult, and the dataset was small, so I tried to mitigate these challenges
with pretraining.
\end{abstract}

\section{Introduction}
In the robotic grasping problem the goal is, given an object, select a grasp
configuration such that the object can be restrained and
manipulated to some desirable end. Finding such configurations is difficult
because of the multi-modal nature of the input and the fact that there can be
more than one suitable grasping location, leaving machines with the task
of determing optimality for predicted configurations.

Some of the earliest reviews of algorithms for grasping \cite{shimoga96,bicchi00},
shows that the premliminary work involved solving unconstrained linear programming
problems using objectives that measure dexterity and grasp quality. A grasp
algorithm in this context is one that is able to calculate the stability or
equilibrium of its grasp, and they are collectively refered too in the
review as \textit{robot grasp synthesis algorithms}.

The review by Sahbani \textit{et al.} \cite{sahbani12}, makes a distinction
between methods that model the kinematics and dynamics of a grasp, like the
synthesis algorithms, and methods that mimic human grasp strategies or learn
from data. The former called the analytic methods and the latter empirical.
They divide the analytical techniques into force closure methods and task
oriented. The authors determine that force closure is able to find stable
grasps but are usually not task oriented, and the task oriented strategies tend
to be computationally intensive. The empirical methods on the other hand can
model task oriented features, but struggle to generalize well to new objects.

Bohg \textit{et al.} \cite{bohg14} observe that grasping methods typically aim
to address the following object types: fully known, familiar, or fully unknown.
When considering the empirical methods, fully known represents objects that
have been seen in the training data before and the grasping problem reduces to
locating the object and applying a similar grasp to those from experience.
The familiar objects are assumed to have some matching characteristics to objects
from the training data, like shape, color, and texture. Familiar objects
introduce a new challenge of calculating similarity between objects so that
the appropriate grasp can be determined. On the other hand, grasping algorithms
will have no experience to utilize when approaching fully known objects. Methods
of this category typically rely on finding structures in the features to
synthesize a grasp.

The focus of this work is to build empirical models for grasp synthesis using
the Cornell Grasp Datset \cite{lenz15}, and evaluating them on familiar and
fully unknown objects. 


supervised learning problem. More recent works use deep learning to detect
possible grasps \cite{lenz15,zhang18,zhou18}, and the dataset from [Saxena]
is used to train the models described in this project. The rest of the
introduction will be used to discuss the analytical, empirical, and deep learning
methods from past works.

\subsection{Analytical Methods}
In Nguyen \textit{et al.} \cite{nguyen86}, the authors define
\textit{force closure grasp}
which occurs if forces applied by the fingers of the end
effector can be balanced against external forces and torques
(the latter is called the wrench force). This work is extended by Ferrari
\textit{et al.} \cite{ferrari92} to produce a \textit{quality criteria}
which is defined as
the ratio between the magnitude of the maximum wrench force to the
applied finger forces. The authors present an algorithm that incorporates the
quality criteria, to find the best grasp, by solving an optimization problem
that minimizes the finger force but maximizes the wrench force for each member
in a set of proposed grasps. In Zhang \textit{et al} \cite{zhang12}, the authors
model various features of an object like surface properties, weight, center
of mass, and weight distribution. These would be potentially useful in
determining the best location for a grasp.

Software exists for simulating the kinematics and dynamics of a robotic
grasp \cite{miller04}, which models the grasp wrench space. It allows the use
of objects with different shapes and surface friction forces, and has a library
of robotic hands with various morphologies. In the simulation, various grasps
can be evaluated before the best one is attempted by a physical robot. Miller
\textit{et al.} \cite{miller03} designed a grasp planning algorithm that
models objects as a
set of shape primitives. Poses were defined for each of
these primitives and grasps
were generated from these poses, to be evaluated by the simulation software in
\cite{miller04}.

\subsection{Empirical Methods}
The empirical methods involve techniques that implement learning algorithms that
model the grasping problem from data. This section will only discuss the
non-deep learning methods, deep learning is discussed in another section.

put something here about kmeans papers \cite{coelho01,piater02}

The simulator Graspit! \cite{miller04} enabled researchers to collect synthetic
data for grasping, one example is by Pelossof \textit{et al.} \cite{pelossof04}.
The authors generated synthetic data by subsampling the parameters from
generated grasps and they trained an
SVM-RBF regressor with the goal of predicting grasp quality.

The work done by Saxena \textit{et al.} \cite{saxena07,saxena08} proposed
algorithms that used visual features to predict grasps, with the distinction from
previous work being that the objects are only partially observable from sets of
2-d images. They extract features that represent edges, textures, and color.
The data was synthetic, constructed using a graphics library called PovRay.
The task is to predict whether a location, which is the center of an image patch,
contains a grasp. The modeling is done using logistic regression, and predicting
the 3-d location is done using a simple probablistic model with naive Bayes
assumptions.

\subsection{Summary of goals}
There are some research papers on the work (early), which are not necessarily
about solving with learning.
The paper associated with
this dataset is also interesting. Hanbo's paper tries a different approach.
There is also pinto and gupta and google which take different approaches
involving learning from attempting grasps. Another approach involves using
techniques from object detection architectures like faster rcnn and ssd.


\section{Data Analysis}
I ran a series of tests, examining the input and targets as I change the task.
I take a look at how I could apply linear regression to the task and see if
I can get any results. I look at the depth data and see how I can extract
features from it. I look at how data augmentation affects the task, and the
clamping as well.

\section{Experiments}
I talk about the hardware I have access too. I trained some networks. I try
to run some ablation experiments with and without features I extracted (and
modification) in data analysis.

\bibliographystyle{plain}
\bibliography{bibliography.bib}
\end{document}
